
\usepackage{esvect,cancel} 
\newcommand{\chapeaumelon}[1]{\stackrel{\Large \frown}{#1}}

%%%%%%%% pour les figures en tikz
\usepackage{tikz}
\usepackage{tkz-tab,tkz-euclide}
\usetkzobj{all}
\usepackage{pgf}
\usetikzlibrary{arrows}
\usetikzlibrary{patterns}  
\definecolor{CyanTikz40}{cmyk}{.4,0,0,0}
\definecolor{CyanTikz20}{cmyk}{.2,0,0,0}

\definecolor{B1prime}      {cmyk}{0.00, 1.00, 0.00, 0.50}
\definecolor{H1prime}      {cmyk}{0.50, 0.00, 1.00, 0.00}

\tikzstyle{general}         =[font=\fontsize{7.5}{9}\selectfont,line width=0.3mm, >=stealth, x=1cm, y=1cm,line cap=round, line join=round]
\tikzstyle{quadrillage}     =[line width=0.3mm, color=CyanTikz40]
\tikzstyle{quadrillageNIV2} =[line width=0.3mm, color=CyanTikz20]
\tikzstyle{quadrillage55}   =[line width=0.3mm, color=CyanTikz40, xstep=0.5, ystep=0.5]
\tikzstyle{cote}            =[line width=0.3mm, <->]
\tikzstyle{epais}           =[line width=0.5mm, line cap=butt]
\tikzstyle{tres epais}      =[line width=0.8mm, line cap=butt]
\tikzstyle{axe}             =[line width=0.3mm, ->, color=Noir, line cap=rect]
\newcommand{\quadrillageSeyes}[2]{%
  \draw[line width=0.3mm, color=A1!10, ystep=0.2, xstep=0.8] #1 grid #2;
  \draw[line width=0.3mm, color=A1!30, xstep=0.8, ystep=0.8] #1 grid #2;
}

% ajouter pour manuel Flo
\newcommand*\circled[1]{\tikz[baseline=(char.base)]{
	\node[shape=circle,draw,inner sep=1pt] (char) {#1};}}

\newcommand{\axeX}[4][0]{%
  \draw[axe] (#2,#1)--(#3,#1);
  \foreach \x in {#4} {\draw (\x,#1) node {\small $+$};
    \draw (\x,#1) node[below] {\small $\numprint{\x}$};
  }%
}
\newcommand{\axeY}[4][0]{%
  \draw[axe] (#1,#2)--(#1,#3);
  \foreach \y in {#4} {\draw (#1, \y) node {\small $+$};
    \draw (#1, \y) node[left] {\small $\numprint{\y}$};
  }%
}
\newcommand{\axeOI}[3][0]{%
  \draw[axe] (#2,#1)--(#3,#1);
  \draw (1,#1) node {\small $+$};
  \draw (1,#1) node[below] {\small $I$};
}
\newcommand{\axeOJ}[3][0]{%
  \draw[axe] (#1,#2)--(#1,#3);
  \draw (#1, 1) node {\small $+$};
  \draw (#1, 1) node[left] {\small $J$};
}
\newcommand{\axeXgraduation}[2][0]{%
  \foreach \x in {#2} {\draw (\x,#1) node {\small $+$};}%
}
\newcommand{\axeYgraduation}[2][0]{%
  \foreach \y in {#2} {\draw (#1, \y) node {\small $+$};}%
}
\newcommand{\origine}{%
  \draw (0,0) node[below left] {\small $0$};
}
\newcommand{\origineO}{%
  \draw (0,0) node[below left] {$O$};
}
\newcommand{\point}[4]{%
  \draw (#1,#2) node[#4] {$#3$};
}
\newcommand{\pointGraphique}[4]{%
  \draw (#1,#2) node[#4] {$#3$};
  \draw (#1,#2) node {$+$};
}
\newcommand{\pointFigure}[4]{
  \draw (#1,#2) node[#4] {$#3$};
  \draw (#1,#2) node {$\times$};
}
\newcommand{\pointC}[3]{
  \draw (#1) node[#3] {$#2$};
}
\newcommand{\pointCGraphique}[3]{
  \draw (#1) node[#3] {$#2$};
  \draw (#1) node {$+$};
}
\newcommand{\pointCFigure}[3]{
  \draw (#1) node[#3] {$#2$};
  \draw (#1) node {$\times$};
}

