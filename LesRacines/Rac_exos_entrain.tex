
\serie{Egalité de vecteurs}

\begin{exercice}
Complétez les pointillés :
\begin{multicols}{2}
\begin{enumerate}
\item $\sqrt{400}=...$
\item $\sqrt{25}-\sqrt{16}=...$
\item $-\sqrt{4}=...$
\item $\sqrt{...}=1$
\item $\sqrt{25-16}=...$
\item $\sqrt{-4}...$
\item $\sqrt{\displaystyle\frac{...}{...}}=\displaystyle\frac{11}{6}$
\end{enumerate}
\end{multicols}

\end{exercice}

\begin{exercice}
Simplifiez au maximum les expressions en utilisant les règles sur les racines :
\begin{enumerate}
\item $A=\sqrt{18}$
\item $B=\displaystyle \sqrt{72}$
\item $C=\sqrt{125}$
\item $D=\sqrt{5}-2\sqrt{20}+\sqrt{45}$
\item $E=\sqrt{75}-2\sqrt{48}+3\sqrt{3}$
\item $F=\sqrt{36}-3\sqrt{72}+2\sqrt{98}$
\end{enumerate}
\end{exercice}

\begin{exercice}
Simplifiez au maximum les expressions en utilisant les règles sur les racines :
\begin{enumerate}
\item $A=\displaystyle \frac{\sqrt{24}}{\sqrt{3}}$
\item $B=(\sqrt{5}-2)^2$
\item $C=\displaystyle (1+\sqrt{6})(\sqrt{2}+\sqrt{3})$
\end{enumerate}
\end{exercice}

\begin{exercice}
Simplifiez au maximum les expressions en utilisant les règles sur les racines :
\begin{multicols}{2}
\begin{enumerate}
\item $A=\displaystyle\frac{\sqrt{225}}{\sqrt{80}}$
\item $B=\sqrt{12} \cdot \sqrt{\displaystyle\frac{1}{27}}$\\
\item $C=\sqrt{\displaystyle\frac{7}{2}} \div \sqrt{\displaystyle \dfrac{7}{32}}$
\end{enumerate}
\end{multicols}
\end{exercice}

\begin{exercice}
Simplifiez au maximum les expressions en utilisant les règles sur les racines :
\begin{multicols}{2}
\begin{enumerate}
\item $A=\displaystyle\frac{1}{\sqrt{5}}$
\item $B=\displaystyle\frac{2}{\sqrt{12}}$
\item $C=\displaystyle\frac{3}{2\sqrt{5}}$
\end{enumerate}
\end{multicols}
\end{exercice}

\begin{exercice}
Simplifiez au maximum les expressions en utilisant les règles sur les racines :
\begin{multicols}{2}
\begin{enumerate}
\item $A=\displaystyle\frac{2}{3-\sqrt{2}}$
\item $B=\displaystyle\frac{2}{3+\sqrt{3}}$
\item $C=\displaystyle\frac{\sqrt{3}}{\sqrt{3}-1}$
\item $D=\displaystyle\frac{1}{\sqrt{2}-\sqrt{3}}$
\item $E=\displaystyle\frac{\sqrt{3}-\sqrt{2}}{\sqrt{3}+\sqrt{2}}$
\end{enumerate}
\end{multicols}
\end{exercice}



\begin{exercice}
Simplifiez au maximum les expressions en utilisant les règles sur les racines :
\begin{multicols}{2}
\begin{enumerate}
\item $A=\sqrt{(-5)^2}$
\item $B=\displaystyle\frac{\sqrt{5}}{\sqrt{5}+\sqrt{2}}$
\item $C=(\sqrt{5}-\sqrt{3})^2$
\item $D=\displaystyle\frac{4-2\sqrt{3}}{{2+\sqrt{3}}}$
\item $E=-\sqrt{2^2+3^2}$
\end{enumerate}
\end{multicols}
\end{exercice}

\begin{exercice}
Simplifiez au maximum les expressions suivantes:
\begin{multicols}{2}
\begin{enumerate}
\item $A=\sqrt[3]{1000}$
\item $B=\sqrt[5]{-32}$
\item $C=\sqrt{\sqrt{81}}$
\item $D=\sqrt{0,01}$
\item $E=\sqrt[3]{-27}$
\item $F=\sqrt{\sqrt{144}}$
\end{enumerate}
\end{multicols}
\end{exercice}

\serie{Divers}

\begin{exercice}
On considère un carré de côté $\sqrt{3} + 3$ cm et un rectangle dont les  dimensions  sont:

$\sqrt{72}+ 3\sqrt{6}$ cm  et $\sqrt{2}$ cm .

Démontrer que ce carré et ce rectangle ont la même aire.
\end{exercice}

\begin{exercice}
Soit P le nombre défini par : P = $\left( 2 \sqrt{5}+\sqrt{8}\right) ^2$.
\begin{enumerate}
\item Ecrire P sous la forme $a+b\sqrt{c}$ où $a$, $b$ et $c$ sont des entiers, $c$ le plus petit possible.
\item Quel nombre positif a pour carré $59+30\sqrt{2}$?
\end{enumerate}
\end{exercice}

\begin{exercice}
On considère les nombres suivants:

A = $2\sqrt{27}-2\sqrt{3}+\sqrt{12}$ et 

B = $\sqrt{75}+\sqrt{48}-7\sqrt{3}$.

Montrer en détaillant les calculs que $\dfrac{\text{A}}{\text{B}}$ est un nombre entier.
\end{exercice}

\begin{exercice}
Les nombres suivants sont-ils des rationnels, des entiers relatifs ou des entiers naturels ? Justifier votre réponse.
\begin{multicols}{2}
\begin{enumerate}
\item $A=3\left( \sqrt{2}-1\right)^2 $
\item $B=\sqrt{3^2+5^2}$
\item $C=\sqrt{3^2}+\sqrt{5^2}$
\item $D=\left( 2\sqrt{3}+\sqrt{2}\right) ^2$
\item $E=\displaystyle\dfrac{\left( \sqrt{6}+\sqrt{2}\right) ^2}{2+\sqrt{3}}$
\item $F=\displaystyle\dfrac{10\sqrt{5}-5\sqrt{2}}{2\sqrt{5}+2}$
\end{enumerate}
\end{multicols}
\end{exercice}

