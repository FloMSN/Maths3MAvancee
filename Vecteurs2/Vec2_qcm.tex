

\QCMautoevaluation{Pour chaque question, plusieurs réponses sont
  proposées.  Déterminer celles qui sont correctes.}

\begin{QCM}
\begin{EnonceCommunQCM}
On a placé 6 points dans un repère orthonormé. 

\begin{center}
\begin{tikzpicture}[general, scale=0.7]
 \draw[quadrillage] (-5,-4) grid (5,3);
 \foreach \x/\y/\N/\pos in {-4/-3/E/below left,-2/-1/F/below left,-2/1/C/below left,-1/2/A/below left,2/-1/B/below left,3/1/D/below left}
 {\pointGraphique{\x}{\y}{\N}{\pos}}
 \axeX{-5}{5}{-3,-2,-1,1,2,-4,3,4}
\axeY{-4}{3}{-3,-2,-1,1,2}
\end{tikzpicture}
\end{center}
\end{EnonceCommunQCM}

  \begin{GroupeQCM}
  
  \begin{exercice}Les composantes du vecteur $\vv{AB}$ sont: 
\begin{ChoixQCM}{4}
\item $\left(\begin{array}{c}2\\1\end{array}\right)$
\item $\left(\begin{array}{c}-2\\2\end{array}\right)$
\item $\left(\begin{array}{c}3\\-3\end{array}\right)$
\item $\left(\begin{array}{c}-3\\3\end{array}\right)$
\end{ChoixQCM}
\begin{corrige}
\reponseQCM{c}
\end{corrige}
\end{exercice}

\begin{exercice}Les composantes du vecteur $\vv{CD}$ sont:
\begin{ChoixQCM}{4}
\item $\left(\begin{array}{c}-2\\3\end{array}\right)$
\item $\left(\begin{array}{c}1\\5\end{array}\right)$
\item $\left(\begin{array}{c}0\\5\end{array}\right)$
\item $\left(\begin{array}{c}5\\0\end{array}\right)$
\end{ChoixQCM}
\begin{corrige}
\reponseQCM{d}
\end{corrige}
\end{exercice}

\begin{exercice}Les composantes du point $G$ tel que $\vv{BG}=\vv{FA}$ sont: 
\begin{ChoixQCM}{4}
\item  $(1;3)$
\item  $(3;2)$
\item  $(2;3)$
\item  $(1;-4)$
\end{ChoixQCM}
\begin{corrige}
\reponseQCM{b}
\end{corrige}
\end{exercice}
\begin{exercice}Les composantes du point $H$ tel que $\vv{HE}=\vv{AD}$ sont:
\begin{ChoixQCM}{4}
\item  $(-8;-2)$
\item  $(0;4)$
\item  $(4;-2)$
\item  $(-2;8)$
\end{ChoixQCM}
\begin{corrige}
\reponseQCM{a}
\end{corrige}
\end{exercice}

\begin{exercice}Les composantes du point $I$ tel que $ACBI$ soit un parallélogramme sont:
\begin{ChoixQCM}{4}
\item  $(-1;1)$
\item  $(3;0)$
\item  $(1;-2)$
\item  $(-5;4)$
\end{ChoixQCM}
\end{exercice}
\begin{corrige}
\reponseQCM{b}
\end{corrige}

    \begin{exercice}
      La norme du vecteur $\vv{AB}$ est:
      \begin{ChoixQCM}{4}
      \item $\sqrt{5}$
      \item $0$
      \item $3\sqrt{2}$
      \item $\sqrt{18}$
      \end{ChoixQCM}
\begin{corrige}
     \reponseQCM{cd}
   \end{corrige}
    \end{exercice}
    
        \begin{exercice}
      La norme du vecteur $\vv{FB}$ est:
      \begin{ChoixQCM}{4}
      \item $-4$
      \item 4
      \item 16
      \item 0
      \end{ChoixQCM}
\begin{corrige}
     \reponseQCM{b}
   \end{corrige}
    \end{exercice}
    
        \begin{exercice}
      La norme du vecteur $\vv{FE}$ est:
      \begin{ChoixQCM}{4}
      \item 8
      \item -8
      \item $\sqrt{8}$
      \item $2\sqrt{2}$
      \end{ChoixQCM}
\begin{corrige}
     \reponseQCM{cd} 
   \end{corrige}
    \end{exercice}

\end{GroupeQCM}
\end{QCM}

  