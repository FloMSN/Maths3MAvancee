

\QCMautoevaluation{Pour chaque question, plusieurs réponses sont
  proposées.  Déterminer celles qui sont correctes.}


\begin{QCM}

\begin{GroupeQCM}

\begin{center}
\begin{tikzpicture}[scale=0.7][general]
 \draw (0,0)--(1,3)--(3,3)--(2,0)--cycle;
 \draw (1,3)--(3,0)--(5,-0.3)--(3,3);
 \draw (0,0)node[left] {$F$};
 \draw (1,3)node[above] {$A$};
 \draw (3,3)node[above] {$B$};
 \draw (2,0)node[below] {$C$};
 \draw (5,-0.3)node[below] {$D$};
 \draw (3,0)node[below] {$E$};
 \draw[color=C1] (0.5,1.5)node {{\boldmath $\infty$}};
 \draw[color=C1] (2.5,1.5)node {{\boldmath $\infty$}};
 \draw[color=F1] (2,3)node[rotate=90] {{\boldmath $\approx$}};
 \draw[color=F1] (1,0)node[rotate=90] {{\boldmath $\approx$}};
 \draw[color=F1] (4,-0.1)node[rotate=90] {{\boldmath $\approx$}};
\end{tikzpicture}
\end{center}


\begin{exercice}$\overrightarrow{AB}+\overrightarrow{BD}=\overrightarrow{AD}$
\begin{ChoixQCM}{2}
\item vrai
\item faux
\end{ChoixQCM}
\begin{corrige}
\reponseQCM{a}
\end{corrige}
\end{exercice}

\begin{exercice}$AB+BD=AD$
\begin{ChoixQCM}{2}
\item vrai
\item faux
\end{ChoixQCM}
\begin{corrige}
\reponseQCM{b}
\end{corrige}
\end{exercice}

\begin{exercice}$ABDE$ est un parallélogramme.
\begin{ChoixQCM}{2}
\item vrai
\item faux
\end{ChoixQCM}
\begin{corrige}
\reponseQCM{b}
\end{corrige}
\end{exercice}

\begin{exercice}$FCBA$ est un parallélogramme.
\begin{ChoixQCM}{2}
\item vrai
\item faux
\end{ChoixQCM}
\begin{corrige}
\reponseQCM{a}
\end{corrige}
\end{exercice}


\begin{exercice}$\overrightarrow{AB}=\overrightarrow{CF}$
\begin{ChoixQCM}{2}
\item vrai
\item faux
\end{ChoixQCM}
\begin{corrige}
\reponseQCM{b}
\end{corrige}
\end{exercice}

\begin{exercice}$\overrightarrow{DE}=\overrightarrow{BA}$
\begin{ChoixQCM}{2}
\item vrai
\item faux
\end{ChoixQCM}
\begin{corrige}
\reponseQCM{b}
\end{corrige}
\end{exercice}

\begin{exercice}Une expression plus simple de la somme $\overrightarrow{BC}-\overrightarrow{BA}+2\overrightarrow{CD}-\overrightarrow{AD}$ est
\begin{ChoixQCM}{4}
\item $\overrightarrow{CD}$
\item $\overrightarrow{BD}$
\item $\overrightarrow{0}$
\item Autre réponse
\end{ChoixQCM}
\begin{corrige}
\reponseQCM{a}
\end{corrige}
\end{exercice}

\begin{exercice}Un quadrilatère $IJKL$ est un parallélogramme si et seulement si
\begin{ChoixQCM}{4}
\item $\overrightarrow{IJ}=\overrightarrow{KL}$
\item $\overrightarrow{IJ}=\overrightarrow{LK}$
\item $\overrightarrow{IK}=\overrightarrow{JL}$
\item $\overrightarrow{IL}=\overrightarrow{JK}$
\end{ChoixQCM}
\begin{corrige}
\reponseQCM{b,d}
\end{corrige}
\end{exercice}

\begin{exercice}Le point $I$ est le milieu du segment $[AB]$ si et seulement si
\begin{ChoixQCM}{4}
\item $\overrightarrow{AI}+\overrightarrow{IB}=\overrightarrow{AB}$
\item $\overrightarrow{AI}=\overrightarrow{BI}$
\item $\overrightarrow{AI}+\overrightarrow{IB}=\overrightarrow{0}$
\item $\overrightarrow{IA}+\overrightarrow{IB}=\overrightarrow{0}$
\end{ChoixQCM}
\begin{corrige}
\reponseQCM{d}
\end{corrige}
\end{exercice}

\begin{exercice}Si $\overrightarrow{AC}=3\overrightarrow{AB}$ alors 
\begin{ChoixQCM}{3}
\item $\overrightarrow{BC}=2\overrightarrow{BA}$
\item $\overrightarrow{BC}=2\overrightarrow{AB}$
\item $\overrightarrow{CA}=\dfrac{3}{2}\overrightarrow{CB}$
\end{ChoixQCM}
\begin{corrige}
\reponseQCM{b,c}
\end{corrige}
\end{exercice}



\end{GroupeQCM}
\end{QCM}

  
