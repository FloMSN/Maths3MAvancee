

\QCMautoevaluation{Pour chaque question, plusieurs réponses sont
  proposées.  Déterminer celles qui sont correctes.} 

\begin{QCM}
\begin{GroupeQCM}
\begin{exercice}L'image de 2 par la fonction $f$, définie sur $\mathbb{R}$ par $f(x)=-3x^2+5x-1$ est:
   \begin{ChoixQCM} {3}
   \item $-22$
   \item $-3$
   \item aucune des  réponses
    \end{ChoixQCM}
\end{exercice}
    \begin{corrige}
      \reponseQCM{b}
    \end{corrige}

\begin{exercice}Une pré image de $-5$  par la fonction $f$, définie sur $\mathbb{R}$ par $f(x)=4x -3$ est: 
    \begin{ChoixQCM} {3}
\item       2
\item $-0,5$
\item 0,5
\item $-23$
\item aucune des réponses
    \end{ChoixQCM}
\end{exercice}
    \begin{corrige}
      \reponseQCM{b}
    \end{corrige}

\begin{exercice}On considère la fonction $g$, définie sur $\mathbb{R^*}$ par $g(x)=\dfrac{1}{x}$. L'image de 4 par $g$ est: 
\begin{ChoixQCM} {3}
\item        0,25
\item $-2$
\item $-4$
\item $-0,25$
\item aucune des réponses
    \end{ChoixQCM}
\end{exercice}
    \begin{corrige}
      \reponseQCM{a}
    \end{corrige}

\end{GroupeQCM}
\end{QCM}

\begin{QCM}
\begin{EnonceCommunQCM}
\parbox{0.3\linewidth}{Pour les questions suivantes, on utilise la fonction $f$, définie sur $[-7;5]$, représentée graphiquement ci-contre: }\hfill\parbox{0.59\linewidth}{
\begin{tikzpicture}[general, scale=0.5]
\draw [quadrillage] (-7,-3) grid (7,7);
\axeX{-7}{7,0}{-6,-4,-2,2,4,6}
\axeY{-3}{7}{-2,2,4,6}
\origine
\draw[smooth,samples=100,domain=-6.925:5.08, color=F1,epais] plot(\x,{0.075833*(\x)^3+0.1875*(\x)^2-1.748333*(\x)+1.14});
\fill [color=B1prime] (-6,2) circle (3pt);
\fill [color=B1prime] (-1,3) circle (3pt);
\fill [color=B1prime] (2,-1) circle (3pt);
\fill [color=B1prime] (4,2) circle (3pt);
\fill [color=B1prime] (-3,6) circle (3pt);
\end{tikzpicture}}
\end{EnonceCommunQCM}

\begin{GroupeQCM}
\begin{exercice}Par cette fonction, l'image de $-2$ est: 
 \begin{ChoixQCM}{2}
\item        comprise entre $-7$ et $-6$
\item 4,5
\item comprise entre 4 et 5
\item on ne peut pas savoir
    \end{ChoixQCM}
    \begin{corrige}
      \reponseQCM{c}
    \end{corrige}
\end{exercice}

\begin{exercice}Par cette fonction, 2 est l'image de: 
 \begin{ChoixQCM}{2}
 \item $-1$
 \item 4
 \item $-6$
 \item on ne peut pas savoir
    \end{ChoixQCM}
     \begin{corrige}
      \reponseQCM{bc}
    \end{corrige}
\end{exercice}

\begin{exercice}Par cette  fonction, le nombre 2 a: 
 \begin{ChoixQCM}{2}
\item   exactement deux antécédents
\item exactement 3 antécédents
\item au moins trois antécédents
\item aucune de ces réponses
    \end{ChoixQCM}
     \begin{corrige}
      \reponseQCM{b}
    \end{corrige}
\end{exercice}

\begin{exercice} $f(0)$ est environ égal à:
 \begin{ChoixQCM}{3}
 \item 1
 \item 0,8
 \item 1,15
 \item 1,2
 \item 0,75
 \item aucune de ces valeurs
    \end{ChoixQCM}
     \begin{corrige}
      \reponseQCM{cd}
    \end{corrige}
\end{exercice}
\begin{exercice}La (ou les) valeur(s) éventuelle(s) du réel $x$ pour lesquelles  $f(x)=-1$ sont : 
 \begin{ChoixQCM}{2}
 \item       environ $-6,6$ et 2
 \item exactement 3
 \item seulement 2
 \item aucune de ces réponses
    \end{ChoixQCM}
     \begin{corrige}
      \reponseQCM{a}
    \end{corrige}
\end{exercice}
\end{GroupeQCM}
\end{QCM}




  