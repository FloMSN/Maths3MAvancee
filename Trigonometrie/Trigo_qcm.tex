\QCMautoevaluation{Pour chaque question, plusieurs réponses sont
  proposées.  Déterminer celles qui sont correctes.}

\begin{QCM}
  \begin{GroupeQCM}
    \begin{exercice}
      TGP est un triangle rectangle en P donc
      \begin{ChoixQCM}{4}
      \item $\cos(\widehat{TGP})=\dfrac{GP}{TP}$
      \item $\sin(\widehat{GTP})=\dfrac{GP}{TG}$
      \item $TG^2=TP^2+PG^2$
      \item $\tan(\widehat{GTP})=\dfrac{GP}{TP}$
      \end{ChoixQCM}
\begin{corrige}
     \reponseQCM{bcd}
   \end{corrige}
    \end{exercice}
    
    \begin{exercice}
     $\tan(45$°)$=\dfrac{AB}{7}$ donc 
      \begin{ChoixQCM}{4}
      \item $AB=7 \cdot \tan(45)$
      \item $AB=\dfrac{\tan(45)}{7}$
      \item $AB=\dfrac{7}{\tan(45)}$
      \item $AB \simeq 7$
      \end{ChoixQCM}
\begin{corrige}
     \reponseQCM{a}
   \end{corrige}
    \end{exercice}
    
        \begin{exercice}
     LNT est un triangle rectangle en N tel que TN = 7cm et LN = 5 cm.
On a donc:
      \begin{ChoixQCM}{4}
      \item $\widehat{TLN}=\dfrac{5}{7}$
      \item $\widehat{TLN}=\simeq 54$°
      \item $\tan(\widehat{TLN})=1,4$
      \item $\tan(\widehat{LTN})= 0,7$
      \end{ChoixQCM}
\begin{corrige}
     \reponseQCM{bc}
   \end{corrige}
    \end{exercice}
    
        \begin{exercice}
      QRS est un triangle rectangle en R tel que SQ = 10 et RQ = 8 (en cm). On a donc:
      \begin{ChoixQCM}{4}
      \item $\widehat{RSQ}=53$°
      \item $\widehat{RSQ} \simeq 37$°
      \item $\widehat{RSQ}= 37$°
      \item $\widehat{RSQ} \simeq 53$°
      \end{ChoixQCM}
\begin{corrige}
     \reponseQCM{d}
   \end{corrige}
    \end{exercice}
    
        \begin{exercice}
      Le sinus d'un angle aigu est
      \begin{ChoixQCM}{4}
      \item un nombre quelconque
      \item un nombre supérieur à 1
      \item un rapport de longueurs
      \item compris entre 0 et 1
      \end{ChoixQCM}
\begin{corrige}
     \reponseQCM{cd}
   \end{corrige}
    \end{exercice}


\end{GroupeQCM}
\end{QCM}

  