

\QCMautoevaluation{Pour chaque question, plusieurs réponses sont
  proposées.  Déterminer celles qui sont correctes.} % Est-ce que c'est toujours le cas ?

\begin{QCM}
  \begin{GroupeQCM}
  
    \begin{exercice}
     $(4x+3)+(2x-6)=0$ donc
      \begin{ChoixQCM}{3}
      \item $6x-3=0$
      \item $4x+3=0$ ou $2x-6=0$
      \item $x=0,5$
      \end{ChoixQCM}
\begin{corrige}
     \reponseQCM{ac}
   \end{corrige}
    \end{exercice}
    
     \begin{exercice}
      $5x(x+2)(2x-3)=0$
      \begin{ChoixQCM}{3}
      \item 0 est une solution
      \item $-2$ et $\dfrac{3}{2}$ sont les solutions
      \item $x=0$ ou $x+2=0$ ou $2x-3=0$
      \end{ChoixQCM}
\begin{corrige}
     \reponseQCM{ac}
   \end{corrige}
    \end{exercice}
    
      \begin{exercice}
    La   somme   de   trois   nombres   entiers naturels, impairs et  consécutifs est égale à 495.
      \begin{ChoixQCM}{3}
      \item Ils sont solutions de $6n-3=495$
      \item Les nombres sont 163, 165 et 167.
      \item Ils vérifient $2n-1+2n+1+2n+3=495$
      \end{ChoixQCM}
\begin{corrige}
     \reponseQCM{bc}
   \end{corrige}
    \end{exercice}

  \begin{exercice}
      Le système 
      $\left\lbrace\begin{array}{lll}
	5x+2y&=& 7\\
	-2x+y&=& -10
	\end{array}\right.$
      \begin{ChoixQCM}{3}
      \item admet pour solution 3 et $-4$    
      \item admet pour solution $(3;-4)$
      \item admet une infinité de solutions
      \end{ChoixQCM}
\begin{corrige}
     \reponseQCM{b} % ici deux réponses justes
   \end{corrige}
    \end{exercice}

  \begin{exercice}
       Le système 
      $\left\lbrace\begin{array}{lll}
	3x-2y&=& 4\\
	2x+y&=& -2
	\end{array}\right.$
      \begin{ChoixQCM}{3}
     \item admet pour solution $(0;-2)$
      \item n'a pas de solution
      \item admet une infinité de solutions
      \end{ChoixQCM}
\begin{corrige}
     \reponseQCM{a}
   \end{corrige}
    \end{exercice}
    
  \begin{exercice}
      On a garé des voitures et des deux-roues. Au total, il y a 52 roues et 16 véhicules. Combien y a-t-il de voitures ?
      \begin{ChoixQCM}{3}
      \item 6 voitures
      \item aucune voiture
      \item 10 voitures
      \end{ChoixQCM}
\begin{corrige}
     \reponseQCM{c} 
   \end{corrige}
    \end{exercice}
    

\end{GroupeQCM}
\end{QCM}

  