
\serie{Equations}

\begin{exercice}
Résoudre les équations suivantes :
\begin{enumerate}
\item $-5 \cdot (2x+3)=2 \cdot (-15-5x)$
\item $5x+3=3$
\item $5x+1=5x-1$
\item $\dfrac{2x+3}{3}-\dfrac{7-5x}{4}=\dfrac{7}{2}-\dfrac{x}{3}$
\item $2x+\dfrac{x+2}{2}=1-\dfrac{x}{2}$
\end{enumerate}

\end{exercice}


\begin{exercice}
Joey pense à un nombre, lui ajoute 11, multiplie le tout par 3 et au résultat obtenu il retranche 3. Joey obtient 51. Quel est le nombre de départ ?
\end{exercice}

\begin{exercice}
Cette année l’âge d’Anna  est le triple de celui de Benoit, mais dans 15 ans il n’en 	sera plus que le double. Quels âges ont Anna et Benoit ?
\end{exercice}

\begin{exercice}
On considère un nombre formé de 2 chiffres dont la somme est 9.\\
Si on permute la place des chiffres et que l’on retranche 27, on retrouve le nombre 	initial. Quel est ce nombre ?
\end{exercice}

\begin{exercice}
Trouvez un nombre de 2 chiffres, sachant qu'il est égal au quadruple de la somme de ses chiffres, et que le chiffre des unités dépasse de 3 le chiffre des dizaines.
\end{exercice}

\begin{exercice}
Résoudre les équations suivantes :
\begin{enumerate}
\item $(x-2)^2=9(2-x)$
\item $x^2-7x+6=0$
\item $(x+3)^2=4x^2$
\item $x^2+10x+21+(x+3)^2=0$
\item $x^3-2x^2=3x$
\item $5x^5-4x^4=5x-4$
\end{enumerate}
\end{exercice}

\begin{exercice}
Résoudre les équations suivantes :
\begin{enumerate}
\item $\dfrac{2}{5}x-\dfrac{1}{9}=\dfrac{3}{9}x+\dfrac{4}{5}$
\item $(x+4)(x-2)-(x+4)(1-2x)=0$
\item $(x+10)^2-100=0$
\item $(2x+3)-(x+5)=15$
\item $(2x+3)^2-x^2=0$
\item $8x^3-56x^2-18x+126=0$
\end{enumerate}
\end{exercice}

\serie{Systèmes}

\begin{exercice}
Résoudre les systèmes suivants en utilisant la méthode la plus adaptée:

\begin{enumerate}

 \item $\left\lbrace\begin{array}{lll}
x-y&=& 11\\
2x&=& 3y+25
\end{array}\right.$

\item $\left\lbrace\begin{array}{lll}
 3x-2y&=&22 \\
 5x+3y&=&24
\end{array}\right.$

\item $\left\lbrace\begin{array}{lll}
 7x+4y&=&9 \\
 -2x+3y&=&14
\end{array}\right.$
\end{enumerate}
\end{exercice}


\begin{exercice}
Résoudre les systèmes suivants en utilisant la méthode la plus adaptée:

\begin{enumerate}
\item $\left\lbrace\begin{array}{lll}
 2x-6y&=&10 \\
 -3x+9y&=&-15
\end{array}\right.$

\item $\left\lbrace\begin{array}{lll}
4x+3y&=& 5\\
8x+6y&=& 11
\end{array}\right.$

 \item $\left\lbrace\begin{array}{lll}
\dfrac{x}{3}-\dfrac{5y}{12}&=& -2\\\\
\dfrac{2x}{7}-\dfrac{y}{14}&=& 3
\end{array}\right.$

\end{enumerate}
\end{exercice}

\begin{exercice}
Résoudre les systèmes suivants en utilisant la méthode qui vous semble la plus judicieuse:

\begin{enumerate}
 \item $\left\lbrace\begin{array}{lll}
x-2y&=& -5\\
7x+10y&=& 1
\end{array}\right.$

\item $\left\lbrace\begin{array}{lll}
 5x+5y&=&5 \\
 3x-7y&=&-2
\end{array}\right.$

\item $\left\lbrace\begin{array}{lll}
 5x+6y&=&-2 \\
 10x+3y&=&-7
\end{array}\right.$

\item $\left\lbrace\begin{array}{lll}
 5x+4y&=&13 \\
 2x-7y&=&31
\end{array}\right.$

\end{enumerate}
\end{exercice}

\begin{exercice}
Une tirelire comporte des pièces de 2 CHF et 5 CHF.\\
Sachant qu’il y a 15 pièces en tout et qu'il y a 54 CHF dans la tirelire, déterminer le nombre de pièces de chaque sorte.
\end{exercice}

\begin{exercice}
Il y a 6 ans Alex avait quatre fois l’âge d’Elisa. Dans quatre ans, Alex aura deux fois 	l’âge d’Elisa.\\
 Quels âges ont Alex et Elisa actuellement ?
\end{exercice}

\begin{exercice}
Le périmètre d’un rectangle vaut 76 cm. Si on diminue la largeur de 3 cm et que l’on augmente sa longueur de 1 cm, alors son aire diminuerait de 65 cm$^2$.\\
Quels sont les dimensions de ce rectangle ?
\end{exercice}

\begin{exercice}
Un confiseur prépare deux sortes de boîtes comprenant des petits macarons et des grands. Dans la première boîte, il place dix petits macarons et quatre grands : cette boîte est vendue 7,20 CHF. Dans la seconde boîte, il place cinq petits macarons et six grands : cette boîte est vendue 7,80 CHF.\\
Calculer le prix en francs de chaque sorte de macarons.
\end{exercice}

\begin{exercice}
Alexandra a pour l'instant deux notes en géographie. Une épreuve qui compte triple et une récitation qui compte une fois. Elle a une moyenne provisoire de 4,75. Elle préférerait que la récitation compte triple et l'épreuve une seule fois, car cela lui ferait une moyenne de 5,25.\\
Quelle est sa note d'épreuve ?
\end{exercice}

\begin{exercice}
Trouvez un nombre de 2 chiffres, sachant qu'il est égal au quadruple de la somme de ses chiffres, et que le chiffre des unités dépasse de 3 le chiffre des dizaines.
\end{exercice}
