

\QCMautoevaluation{Pour chaque question, plusieurs réponses sont
  proposées.  Déterminer celles qui sont correctes.} 

\begin{QCM}
  \begin{GroupeQCM} 
  
    \begin{exercice}
      Parmi les nombres suivants, des solutions de l'inéquation
$2x + 7 \leq 3x + 5$ sont...
      \begin{ChoixQCM}{4}
      \item $-1$
      \item 0
      \item 3
      \item 2
      \end{ChoixQCM}
\begin{corrige}
     \reponseQCM{cd} 
   \end{corrige}
    \end{exercice}
    
    \begin{exercice}
      Le nombre 3 est solution de l'inéquation...
      \begin{ChoixQCM}{4}
      \item $3x+7<x-3$
      \item $2x-5\geq 1$
      \item $4x-4>x+1$
      \item $(x+7)^2>80$
      \end{ChoixQCM}
\begin{corrige}
     \reponseQCM{bcd} 
   \end{corrige}
    \end{exercice}
    
    \begin{exercice}
      L'inéquation qui a pour solutions tous les nombres  inférieurs ou égaux à $-2$  est...
      \begin{ChoixQCM}{4}
      \item $3x<-6$
      \item $x+2\geq 4x+8$
      \item $-5x\leq10$
      \item $8\geq x+10$
      \end{ChoixQCM}
\begin{corrige}
     \reponseQCM{bd} 
   \end{corrige}
    \end{exercice}
    
     \begin{exercice}
      $3x+2 \leq 2x + $1 possède exactement les mêmes solutions que...
      \begin{ChoixQCM}{4}
      \item $3x\leq 2x-1$
      \item $2x+1 \geq 3x+2$
      \item $x\leq 1$
      \item $-x \geq -1$
      \end{ChoixQCM}
\begin{corrige}
     \reponseQCM{ab} 
   \end{corrige}
    \end{exercice}
    
    \begin{exercice}
      Un nombre est supérieur ou égal à $-3$ donc...
      \begin{ChoixQCM}{4}
      \item son triple est strictement supérieur 
à $-3$
      \item son opposé est inférieur 
ou égal à 3
      \item son double peut être égal à $-10$
      \item en ajoutant 5, le résultat 
est positif
      \end{ChoixQCM}
\begin{corrige}
     \reponseQCM{bd} 
   \end{corrige}
    \end{exercice}
    
        \begin{exercice}
      L'inéquation $2x + 5 \leq 2x + 6$...
      \begin{ChoixQCM}{4}
      \item n'a pas 
de solution
      \item admet 7 comme solution
      \item a une infinité de solutions
      \item admet tout nombre positif comme solution
      \end{ChoixQCM}
\begin{corrige}
     \reponseQCM{bcd} 
   \end{corrige}
    \end{exercice}


\end{GroupeQCM}
\end{QCM}

  