
\serie{Inégalités}

\begin{exercice}[]
Complétez le tableau suivant :

\begin{center}
\renewcommand{\arraystretch}{2}
\begin{tabular}{|>{\centering\arraybackslash}p{1.8cm}|>{\centering\arraybackslash}p{2.9cm}|>{\centering\arraybackslash}p{2.1cm}|}
\hline
\rowcolor[gray]{0.8}Intervalle & Représentation graphique & Inégalités\\\hline
$[-4;8]$&&\\\hline
&&$x<7$\\\hline
$]-\infty;3]$&&\\\hline
$]0;7]$&&\\\hline
&&$x\geq -1$\\\hline
&&$2>x\geq -4$\\\hline
\end{tabular}
\end{center}
\end{exercice}

\begin{exercice}[]
Sachant que $a$ est un nombre tel que $a<3$, compléter par une inégalité:
\setlength{\columnseprule}{0pt}
\begin{multicols}{2}[\raggedcolumns]
\begin{enumerate}
\item $a+3 ...$
\item $a-3 ...$
\item $3a ...$
\item $-3a ...$
\item $3a-\pi ...$
\item $-3a+3 ...$
\end{enumerate}
\end{multicols}
\end{exercice}

\begin{exercice}[]
Soient $x$ et $y$ deux nombres réels tels que\\
$-3,5<x<-3,4$ et $2,5<y<2,6$, encadrer les nombres suivants:
\setlength{\columnseprule}{0pt}
\begin{multicols}{2}[\raggedcolumns]
\begin{enumerate}
\item $4y+3$
\item $\dfrac{1}{4y+3}$
\item $7-3y$
\item $-xy$
\item $xy$
\end{enumerate}
\end{multicols}
\end{exercice}

\serie{Inéquations}
\begin{exercice}[]
Résoudre les inéquations suivantes :
\setlength{\columnseprule}{0pt}
\begin{multicols}{2}[\raggedcolumns]
\begin{enumerate}
\item $2x-4 \leq x-5$
\item ${2} + x \geq 3 x - 4$
\item ${2} x + {2} < {2} x - {4}$
\item $\dfrac{x -  {4}}{2} \leq  x - {1}$
\end{enumerate}
\end{multicols}
\end{exercice}

\begin{exercice}[]
Résoudre les inéquations suivantes :
\setlength{\columnseprule}{0pt}
\begin{multicols}{2}[\raggedcolumns]
\begin{enumerate}
\item ${2} (x - {1})  \geq  {2} x - {3}$
\item $x - \dfrac{1}{2} > {3} x - {4}$
\item $\dfrac{x} {5} + \dfrac{3}{10}   \leq {2}x$
\item $\dfrac{3 x - 14}{12} + \dfrac{3 x - 2}{4} > \dfrac{2 x - 1}{3}$
\end{enumerate}
\end{multicols}
\end{exercice}

\begin{exercice}[]
Résoudre les inéquations suivantes:
\setlength{\columnseprule}{0pt}
\begin{multicols}{2}[\raggedcolumns]
\begin{enumerate}
\item $3(x+1)-x\leq1+2(1+x)$
\item $-\dfrac{3}{5}x-6 < - \dfrac{2}{5}x+7$
\item $2x-\dfrac{2}{3}<2-\dfrac{x+1}{3}$ \\
\item $\dfrac{1}{5}-\dfrac{4}{3}\geq2\left( 1-\dfrac{5}{6}x\right) $
\end{enumerate}
\end{multicols}
\end{exercice}

\serie{Problèmes}
\begin{center}
\textit{Résoudre les problèmes suivants par une mise en inéquation.}
\end{center}
\begin{exercice}[]
Clément a 32 ans et Lucie a 5 ans.\\
Pendant combien d’années l’âge de Clément sera-t-il supérieur au quadruple de celui de Lucie ?
\end{exercice}

\begin{exercice}[]
Une session de karting coûte 55 CHF. Mais le prix passe à 11 CHF la session si vous 	achetez une carte d’abonnement à 900 CHF.\\
A partir de combien de session cet abonnement est-il avantageux ?
\end{exercice}

\begin{exercice}[]
Une famille espère économiser 250 CHF par an en récupérant de l’eau de pluie dans 	une citerne. Au bout de combien d’années les économies réalisées pourront-elles 	compenser l’achat de la citerne, qui coûte 910 CHF ?
\end{exercice}

\begin{exercice}[]
Maïa, nouvelle adhérente d’un club de squash, étudie les deux tarifs proposés :
\begin{description}
\item Tarif A : 55 CHF la séance;
\item Tarif B : achat d’une carte privilège annuelle de 400 CHF, donnant droit au tarif réduit de 40 CHF.
\end{description}
Pour combien de séances est-il plus avantageux d’acheter la carte privilège ?
\end{exercice}

\begin{exercice}[]
Un aller-retour en train entre deux villes coûte 80 CHF. Avec un abonnement annuel à 442 CHF, on bénéficie d’une réduction de $50\%$ sur le prix d’un aller-retour.\\
Quelle formule doit-on choisir en fonction du nombre de voyages effectués ?
\end{exercice}
	
\begin{exercice}[]
La somme de trois entiers consécutifs est comprise entre 367 et 372.\\ 
Quels sont ces entiers ?
\end{exercice}
	
\begin{exercice}[]
La largeur d’un terrain rectangulaire est égale à la moitié de sa longueur.\\
Ce terrain est entouré d’une allée de 1 m de large.\\
On sait que l’aire de l’allée est comprise entre 112 m$^{2}$ et 208 m$^{2}$.\\
Encadrez le plus précisément possible la largeur de ce terrain.\\\\\\\\\\\\
\end{exercice}

\begin{exercice}[]
Un bureau de recherche emploie 27 informaticiens et 15 mathématiciens. 

On envisage d’embaucher le même nombre $x$  d’informaticiens et de mathématiciens.

Combien faut-il embaucher de spécialistes de chaque sorte pour que le nombre de mathématiciens soit au moins égal aux deux tiers du nombre d’informaticiens ?\\
\end{exercice}

\begin{exercice}[]
La longueur d’un rectangle dépasse de 7 dm sa largeur. 

On sait que son périmètre est compris entre 20 dm et 26 dm. 

Que peut-on dire au sujet de sa largeur ?

\end{exercice}